\documentclass{article}
\usepackage{graphicx} % Required for inserting images
\graphicspath{{img/}}
\usepackage[spanish]{babel}

\title{LLL: Long Live Linux}
\author{David Elias González García, Cynthia Berenice }
\date{March 2023}

\begin{document}

\maketitle
\tableofcontents
\pagebreak
\section{Introducción}
La historia de la computadora se remonta a varios siglos atrás, desde la creación de dispositivos mecánicos para el cálculo hasta la aparición de las modernas computadoras digitales.\\

Uno de los primeros dispositivos mecánicos fue la "ábaco", utilizado por los antiguos griegos y romanos para realizar operaciones matemáticas simples. Luego, en el siglo XVII, se crearon las "calculadoras mecánicas", como la "Pascalina" inventada por Blaise Pascal y la "Máquina de Diferencias" de Charles Babbage.\\

A principios del siglo XX, se desarrollaron las "máquinas de tabulación" para procesar grandes cantidades de datos. Una de las más famosas fue la "Tabuladora Hollerith", creada por Herman Hollerith para el censo de 1890 en Estados Unidos.\\

La verdadera revolución de las computadoras comenzó en la década de 1940, cuando se crearon las primeras computadoras electrónicas. Una de las más importantes fue la "ENIAC" (Electronic Numerical Integrator and Computer), creada por John Mauchly y J. Presper Eckert en 1945.\\

Desde entonces, la tecnología de las computadoras ha avanzado a pasos agigantados, desde las grandes computadoras de la década de 1950 y 1960 hasta las pequeñas computadoras personales de hoy en día y la tecnología en la nube. La capacidad de procesamiento, el almacenamiento de datos y la velocidad de procesamiento han aumentado exponencialmente a lo largo de los años, permitiendo el desarrollo de nuevas tecnologías y aplicaciones en prácticamente todas las áreas de la vida moderna. \\

Entonces, una computadora es un dispositivo electrónico que procesa y almacena datos. En términos generales, las computadoras pueden realizar las siguientes funciones: \\
\begin{itemize}
    \item Procesamiento de datos: una computadora puede procesar grandes cantidades  de información a una velocidad mucho mayor que la que puede manejar una persona. Esto incluye operaciones matemáticas, lógicas y de comparación, así como la ejecución de programas y aplicaciones.
    \item Almacenamiento de datos: una computadora puede almacenar grandes cantidades de información, incluyendo documentos, imágenes, videos, música y otros tipos de datos. Los datos se pueden guardar en diferentes dispositivos de almacenamiento, como discos duros, unidades flash USB y discos ópticos.
    \item Comunicación: una computadora puede comunicarse con otros dispositivos y con internet a través de una conexión de red. Esto permite compartir información y recursos, como archivos, impresoras y conexiones a internet.
    \item Control de dispositivos: una computadora puede controlar otros dispositivos, como impresoras, escáneres y cámaras, para realizar tareas específicas.
    \item Entretenimiento: una computadora puede reproducir música, películas y juegos para proporcionar entretenimiento.\\
\end{itemize}
        
        
Estas son solo algunas de las funciones básicas que una computadora puede realizar. Con el avance de la tecnología, las computadoras también están siendo utilizadas en una amplia variedad de campos, como la investigación científica, el diseño gráfico, la ingeniería, la medicina y muchos otros.\\

Las computadores utilizan sistemas operativos para funcionar, pero, ¿Qué es un sistema operativo?.
Un sistema operativo es un software que actúa como intermediario entre el hardware de una computadora y los programas de aplicación que se ejecutan en ella. Es el encargado de gestionar los recursos de hardware y software de la computadora, proporcionando una interfaz entre el usuario y la máquina. Los sistemas operativos realizan tareas como la gestión de archivos y directorios, el control de acceso de usuarios, la gestión de la memoria y la asignación de recursos de hardware, la administración de procesos y servicios, entre otras funciones esenciales. Algunos ejemplos de sistemas operativos incluyen Windows, macOS, Linux, Android e iOS.\\

En este caso, hablaremos de GNU/Linux: La historia de GNU/Linux se remonta a principios de la década de 1980, cuando Richard Stallman, fundador del proyecto GNU (acrónimo de "GNU's Not Unix"), comenzó a trabajar en un sistema operativo libre y de código abierto que pudiera ser utilizado por cualquier persona sin restricciones legales o económicas.\\

En 1991, Linus Torvalds, un estudiante finlandés de informática, creó un núcleo de sistema operativo llamado Linux, que se basaba en el sistema operativo Unix. Torvalds hizo que su código fuera de código abierto y disponible para su uso y modificación, lo que permitió que la comunidad de programadores de todo el mundo contribuyera a su desarrollo.\\

La combinación del núcleo de Linux con el software libre de GNU permitió la creación del sistema operativo GNU/Linux, que se ha convertido en una alternativa popular y de uso extendido a los sistemas operativos propietarios de Microsoft Windows y Apple MacOS.\\

Desde entonces, la comunidad de programadores de GNU/Linux ha desarrollado una amplia gama de software libre y de código abierto, incluyendo aplicaciones de oficina, navegadores web, servidores, sistemas de gestión de bases de datos y mucho más. La filosofía del software libre ha permitido que GNU/Linux se convierta en una plataforma popular y de alta calidad para empresas, organizaciones sin fines de lucro y usuarios individuales de todo el mundo.\\

En particular, nos interesa la forma en cómo interactuamos con la computadora, cómo nos comunicamos y le damos instrucciones a ella. Lo cual lo hacemos mediante una interfaz de usuario (UI, por sus siglas en inglés) que es el medio por el cual los usuarios interactúan con un sistema operativo, aplicación o dispositivo. Una interfaz puede ser gráfica (GUI), en la que se utilizan iconos, ventanas y menús, o de línea de comandos (CLI), en la que se utilizan comandos escritos.\\

La línea de comandos es una forma de interactuar con una computadora mediante la introducción de comandos de texto en una interfaz de línea de comandos. Estos comandos se escriben en un intérprete de comandos (shell), que es un programa que acepta comandos y los ejecuta en el sistema operativo subyacente. La CLI se utiliza a menudo en sistemas operativos Unix y Linux, así como en otras aplicaciones de línea de comandos, como Git y Docker.\\

Entonces, nos comunicamos con la computadora mediante la línea de comandos, pero necesitamos una interfaz de usuario. Bueno, para eso utilizamos la terminal.\\

Una terminal es un programa o aplicación que permite a los usuarios interactuar con un sistema operativo o una aplicación mediante líneas de comandos o texto. La terminal suele proporcionar una interfaz de usuario en la que se pueden ingresar comandos, que luego se transmiten al sistema operativo o la aplicación para su ejecución.\\

En el contexto de los sistemas operativos tipo Unix, la terminal se refiere a menudo al emulador de terminal, que es un programa que simula una consola o terminal física y proporciona una ventana en la que los usuarios pueden ingresar comandos y recibir resultados. La terminal también puede utilizarse para ejecutar programas de shell, editar archivos de texto y ejecutar otras tareas de administración del sistema.\\

En resumen, la terminal es una herramienta muy útil para los usuarios avanzados y administradores de sistemas, ya que permite interactuar con el sistema operativo y las aplicaciones de manera más eficiente y flexible que a través de una interfaz gráfica de usuario. Por lo tanto, trabajaremos el desarrollo de una terminal en este proyecto.\\

\section{Desarrollo}

Para empezar a desarrolar comandos en la terminal, se necesita saber c\'omo hacer uno, qu\'e es una variable de entorno, qu\'e es una shell, c\'omo se puede hacer un bash script.\\

Una l\'inea de comandos es una SHELL, la cual, es un programa que toma comandos y los pasa al sistema operativo para procesarlos (interpreta los comandos y le dice al SO qu\'e hacer). Existen varias, como la Power Shell (de Windows), Bourne Shell, C Shell, y para nuestro caso, la que se utilizar\'a ser\'a Bash Shell (de Linux).\\

Para crear comandos, es necesario saber el concepto y para que nos van a servir las variables de entorno. Entonces, una variable de entorno es una variable global que se encuentra en el sistema operativo y puede ser utilizada por procesos en ejecución. Estas variables contienen información sobre el sistema operativo, el usuario actual, la sesión de inicio, la configuración de red, entre otros aspectos.\\

En sistemas Unix y Linux, las variables de entorno se definen como una cadena de texto que se asocia con un valor. Estas variables se utilizan comúnmente para definir parámetros de configuración en el sistema operativo o en aplicaciones específicas. Por ejemplo, la variable de entorno "PATH" se utiliza para especificar los directorios donde se encuentran los ejecutables del sistema, lo que permite que se puedan ejecutar desde cualquier directorio en la línea de comandos.\\

Las variables de entorno también pueden ser creadas o modificadas por el usuario, lo que puede ser útil para personalizar la configuración del sistema o para proporcionar información adicional a las aplicaciones en ejecución.\\



\\

\begin{tabular}{c|c|c}
     &  \\cuac ekis de & interesante \\ & cuyac
     & cuacsdfg\\
     & ooorale
\end{tabular}

\end{document}
